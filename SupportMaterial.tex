%!TEX root = Slic3r-Manual.tex
\section{Mati\`ere de Support} % (fold)
\label{sec:support}
\index{support material}
\index{mati\`ere de support}

En g\'en\'eral, la plupart des mod\`eles 3D seront imprim\'es avec des parties en surplomb jusqu'\`a une certaine inclinaison. L'angle est d\'etermin\'e par plusieurs facteurs, notamment la hauteur de la couche et la largeur d'extrusion, et est g\'en\'eralement autour de 45 °. Pour les mod\`eles avec de plus grands surplombs une structure de support peut \^etre imprim\'e en dessous. Cela engage l'utilisation de plus de mati\`ere, plus de temps d'impression, et un nettoyage apr\`es l'impression.

\begin{figure}[H]
\centering
\includegraphics[keepaspectratio=true,width=1\textwidth]{expertmode/support/advanced_support.png}
\caption{Param\`etres de support.}
\label{fig:advanced_support}
\end{figure}
\index{Print Settings!Support material!Generate support material}
\index{Param\`etres d'Impression!Mati\`ere de Support!G\'en\'erer un support}
\index{Print Settings!Support material!Overhang threshold}
\index{Param\`etres d'Impression!Mati\`ere de Support!Seuil de porte \`a faux}
\index{Print Settings!Support material!Enforce support}
\index{Param\`etres d'Impression!Mati\`ere de Support!Appliquer le support}

La premi\`ere chose \`a faire est d'activer l'option de mati\`ere de support en cochant la case \texttt{Generate support material} (G\'en\'erer un support).  Mettre \`a z\'ero le param\`etre \texttt{Overhang threshold} (Seuil de porte \`a faux) indique \`a Slic3r de d\'etecter les lieux o\`u apporter un soutien automatiquement, sinon l'angle indiqu\'e sera utilis\'e.  La g\'en\'eration de support est un sujet relativement complexe, et il y a plusieurs aspects qui d\'eterminent le soutien optimal, il est fortement recommand\'e de fixer le seuil \`a z\'ero et permettre Slic3r de d\'eterminer le soutien n\'ecessaire.

Les petits mod\`eles, et ceux avec de petites empreintes \`a la base, peuvent parfois se briser ou se d\'etacher du lit.  Pour cette raison le param\`etre \texttt{Enforce support} (Appliquer le support) produira des structures de support \`a imprimer pour le nombre donn\'e de couches, ind\'ependamment de la valeur de seuil d'angle.

Pour d\'emontrer les modes de remplissage le mod\`ele minimug a \'et\'e inclin\'e de 45 ° le long de l'axe x, comme repr\'esent\'e sur la figure \ref{fig:support_minimug_45deg}.
\index{Print Settings!Support material!Pattern}
\index{Param\`etres d'Impression!Mati\`ere de Support!Motif}

\begin{figure}[H]
\centering
\includegraphics[keepaspectratio=true,width=0.75\textwidth]{expertmode/support/support_minimug_45deg.png}
\caption{Mod\`ele Minimug, inclin\'e \`a 45°.}
\label{fig:support_minimug_45deg}
\end{figure}

Comme avec le remplissage, il existe plusieurs motifs disponibles pour la structure de support.

\begin{figure}[H]
\centering
\includegraphics[keepaspectratio=true,width=0.2\textwidth]{expertmode/support/support_pattern_rectlinear.png}
\caption{Motif de support: Rectiligne}
\label{fig:support_pattern_rectlinear}
\end{figure}

\begin{figure}[H]
\centering
\includegraphics[keepaspectratio=true,width=0.2\textwidth]{expertmode/support/support_pattern_rectlinear_grid.png}
\caption{Motif de support: Grille Rectiligne}
\label{fig:support_pattern_rectlinear_grid}
\end{figure}

\begin{figure}[H]
\centering
\includegraphics[keepaspectratio=true,width=0.2\textwidth]{expertmode/support/support_pattern_honeycomb.png}
\caption{Motif de support: Nid d'Abeille}
\label{fig:support_pattern_honeycomb}
\end{figure}
\index{Print Settings!Support material!Pattern Spacing}
\index{Param\`etres d'Impression!Mati\`ere de Support!Espacement du Motif}

\index{Print Settings!Support material!Pattern Angle}
\index{Param\`etres d'Impression!Mati\`ere de Support!Angle du Motif}

\texttt{Pattern Spacing} (Espacement du Motif) d\'etermine la distance entre les lignes de support, et est comparable \`a la densit\'e de remplissage en plus d'\^etre d\'efinie seulement en mm. Si vous changez cet attribut tenez compte de la largeur de l'extrusion du support et de la quantit\'e de mati\`ere de support qui adh\`ere \`a l'objet.

Il faut prendre soin de choisir un motif de support qui correspond au mod\`ele, o\`u le support se fixe perpendiculairement \`a la paroi de l'objet, plut\^ot que parall\`element, de sorte qu'il sera facile \`a retirer.  Si la structure de support court le long de la longueur d'une paroi alors le param\`etre \texttt{Pattern Angle} (Angle du Motif) permet la rotation de la direction des lignes de support

\begin{figure}[H]
\centering
\includegraphics[keepaspectratio=true,width=0.2\textwidth]{expertmode/support/support_pattern_rectlinear_rotated.png}
\caption{Exemple de motif tourn\'e \`a 45°.}
\label{fig:support_pattern_rectlinear_rotated}
\end{figure}


%TODO: Interface layers.


% section support (end)
