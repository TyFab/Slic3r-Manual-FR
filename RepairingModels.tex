%!TEX root = Slic3r-Manual.tex

Si le maillage 3D d\'ecrit dans le mod\`ele contient des trous, ou les bords ne sont pas align\'es (connu comme \'etant non-manifold), alors Slic3r peut avoir des probl\`emes de traitement . Slic3r va tenter de r\'esoudre les probl\`emes, s'il le peut, mais certains probl\`emes sont hors de sa port\'ee. Si l'application indique que le mod\`ele ne peut pas \^etre tranch\'e correctement alors il y a plusieurs options disponibles, et celles d\'ecrites ici sont toutes libres au moment de l'\'ecriture.

%%% CONFIGURATION TUNING %%%
{\input{WorkingWithModelsNetfabb}}
%%% END CONFIGURATION TUNING %%%

\paragraph{FreeCAD} % (fold)
\label{par:freecad}
\index{FreeCAD}

Freecad\footnote{\url{http://sourceforge.net/projects/free-cad}} est un logiciel de CAO, complet et gratuit, qui est livr\'e avec un module de maillage, dans lequel on peut effectuer les r\'eparations d'erreur dans les mod\`eles. Les \'etapes suivantes d\'ecrivent comment un probl\`eme dans un fichier de mod\`ele peut \^etre analys\'e et r\'epar\'e.

\begin{figure}[H]
\centering
\includegraphics[keepaspectratio=true,width=0.75\textwidth]{working_with_models/freecad_part_repair.png}
\caption{R\'eparation avec FreeCAD.}
\label{fig:freecad_part_repair}
\end{figure}

\begin{itemize}
	\item Lancer FreeCAD et \`a partir la page d'accueil choisir \texttt{Working with Meshes}.
	\item Chargez le mod\`ele en le faisant glisser sur l'espace de travail ou par l'interm\'ediaire du menu \texttt{File}.  Un petit message dans le coin en bas \`a gauche indique si le mod\`ele semble avoir des probl\`emes.
	\item Dans le menu choisissez \texttt{Meshes->Analyze->Evaluate \& Repair mesh} pour faire appara\^itre la bo\^ite de dialogue des options de r\'eparation.
	\item Dans la bo\^ite de dialogue choisir la maille charg\'ee, puis effectuer chaque analyse soit en cliquant sur le bouton \texttt{Analyze} par type de probl\`eme, ou s\'electionnez \texttt{Repetitive Repair} en bas pour effectuer tous les contr\^oles. Si un probl\`eme  correspondant est d\'etect\'e le bouton \texttt{Repair} devient actif.
	\item Pour chaque r\'eparation souhait\'e frapper le bouton \texttt{Repair}.
	\item Il est important d'examiner l'effet que le script de r\'eparation a apport\'e au mod\`ele.  Il se peut que le script produise des dommages dans le fichier, plut\^ot que de le r\'eparer, par exemple en retirant des triangles importants.
	\item Exporter le mod\`ele r\'epar\'e par le menu \texttt{Export} ou le menu contextuel.
\end{itemize}
% paragraph freecad (end)
