%!TEX root = Slic3r-Manual.tex

\section{"La" Première Couche}
\label{sec:the_important_first_layer}
\index{First Layer}
\index{Première Couche}

Avant de se lancer tête baissée dans dans la production de la première impression, il est intéressant de s'arrêter pour parler de l'importance d'obtenir une première couche parfaite. Comme beaucoup l'ont constaté par tâtonnements, si la première couche n'est pas la meilleure, cela peut alors conduire à un échec complet, des parties se détachant, et des déformations. Il existe plusieures techniques et recommandations, dont on peut tenir compte afin de minimiser le risque que cela se produise.

\paragraph{Le lit à niveau} % (fold)
\label{par:level_bed}
Avoir un lit de niveau est essentiel. Si la distance entre l'extrémité de la buse et le lit diffère de même quelques microns, il se peut que la matière ne soit pas étendue sur le lit (parce que la buse est trop proche et racle le lit), ou que de la matière se trouvant trop éloignée du lit, n'adhére pas correctement.
% paragraph level_bed (end)

\paragraph{Température plus élevée.} % (fold)
\label{par:higher_temperature}
La tête chauffante et le lit, s'il est chauffé, peuvent être surchauffés pour la première couche, ceci diminue la viscosité de la matière en cours d'impression.  En règle générale, un supplément de 5 ° est recommandé.
% paragraph higher_temperature (end)

\paragraph{Des vitesses inférieures.} % (fold)
\label{par:lower_speeds}
Ralentir l'extrudeuse pour la première couche réduit les efforts appliqués à la matière fondue à la sortie, ce qui réduit les chances d'être trop étirées et de ne pas adhérer correctement. 30\% ou 50\% de la vitesse normale est recommandée.
% paragraph lower_speeds (end)

\paragraph{Taux d'extrusion correctement calibré.} % (fold)
\label{par:correct_extrusion_settings}
Si trop de matière est extrudé alors la buse peut glisser par dessus lors du deuxième passage, en la soulevant par rapport au lit (en particulier si le matériau a refroidi). Trop peu de matière peut faire que la première couche se détache plus tard lors de l'impression, conduisant soit à arrachements ou des déformations. Pour ces raisons, il est important d'avoir un taux d'extrusion bien calibré tel que recommandé au §\ref{calibration}).
% paragraph correct_extrusion_settings (end)

\paragraph{La hauteur de la première couche.} % (fold)
\label{par:first_layer_height}
Une couche épaisse fournira plus de débit, et par conséquent plus de chaleur, ce qui permet à l'extrusion de mieux adhérer au lit. Elle donne aussi l'avantage d'apporter plus de tolérance pour la planéité du lit. Il est recommandé d'augmenter la hauteur de la première couche pour correspondre au diamètre de la buse, par exemple, une première hauteur de la couche de 0,35 mm pour une buse 0.35mm.
Remarque: La hauteur de la première couche est automatiquement réglée de cette façon en mode simple.
% paragraph first_layer_height (end)

\paragraph{Plus grossse largeur d'extrusion.} % (fold)
\label{par:wider_extrusion_width}
Plus il y a de matière à toucher le lit, plus l'objet adhère au lit, ceci peut être obtenu en augmentant la largeur de l'extrusion de la première couche, soit par un pourcentage ou une quantité fixée. Les espaces entre les extrusions sont ajustés en conséquence.

Une valeur d'environ 200 \% est généralement recommandée, mais il faut noter que la valeur est calculée à partir de la hauteur de la couche et donc la valeur ne doit être réglée que si la hauteur de la couche est la plus élevée possible. Par exemple, si la hauteur de la couche est de 0,1 mm, et que la largeur de l'extrusion est réglée à 200 \%, alors la largeur réelle extrudé sera seulement de 0,2 mm, ce qui est plus petite que la buse. Cela risque de provoquer un mauvais écoulement et conduire à une impression ratée. Il est donc fortement recommandé de combiner la hauteur de la première couche, recommandée ci-dessus avec celle-ci. Régler la hauteur de la première couche à 0,35 mm et la première largeur d'extrusion à 200 \% se traduirait par une belle grosse extrusion 0,65 mm de large.
% paragraph wider_extrusion_width (end)

\paragraph{Matériau du lit.} % (fold)
\label{par:bed_material}
Plusieurs solutions existent pour le matériel à utiliser pour le lit, et la préparation de la surface peut considèrablement améliorer l'adhérence de la première couche.

Le PLA est plus tolérant et fonctionne bien sur le PET, Kapton, ou ruban adhèsif de peintre bleu.


L'ABS a généralement besoin de plus d'attentions et, s'il s'imprime bien sur PET et Kapton, on rapporte que les gens ont de bon résultats en appliquant de la laque sur le lit avant de l'imprimer. D'autres ont signalé qu'une solution d'ABS (fabriqué à partir de la dissolution de morceaux d'ABS dans de l'acétone) finement appliquée peut également augmenter l'adhèrence.
% paragraph bed_material (end)

\paragraph{Aucun refroidissement.} % (fold)
\label{par:no_cooling}
Directement lié à ce qui précède, il n'est pas logique d'augmenter la température de la première couche et avoir un ventilateur ou un autre mécanisme de refroidissement en fonctionnement. Garder le ventilateur éteint pendant les quelques premières couches est généralement recommandé.
% paragraph no_cooling (end)
